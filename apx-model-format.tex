\section{Model data format}\label{sec:model-data-format}

Because the model data does not need to be very concise, MessagePack has been
chosen as the data storage format for the models.
The file contains the model type (acids or quality scores), context specifier
type (described in \Cref{subsec:contexts}), list of contexts (symbol
probabilities), and map of context specifiers to context indices.
Precisely, the data stored in such MessagePack file corresponds to a given
JSON file:

\colorlet{punct}{red!60!black}
\definecolor{background}{HTML}{EEEEEE}
\definecolor{delim}{RGB}{20,105,176}
\colorlet{numb}{magenta!60!black}

\lstdefinelanguage{json}{
    basicstyle=\normalfont\ttfamily,
    numberstyle=\scriptsize,
    commentstyle=\color{gray},
    stepnumber=1,
    numbersep=8pt,
    showstringspaces=false,
    breaklines=true,
    frame=single,
    comment=[l]{//},
    backgroundcolor=\color{background},
    literate=
    *{0}{{{\color{numb}0}}}{1}
        {1}{{{\color{numb}1}}}{1}
        {2}{{{\color{numb}2}}}{1}
        {3}{{{\color{numb}3}}}{1}
        {4}{{{\color{numb}4}}}{1}
        {5}{{{\color{numb}5}}}{1}
        {6}{{{\color{numb}6}}}{1}
        {7}{{{\color{numb}7}}}{1}
        {8}{{{\color{numb}8}}}{1}
        {9}{{{\color{numb}9}}}{1}
        {:}{{{\color{punct}{:}}}}{1}
        {,}{{{\color{punct}{,}}}}{1}
        {\{}{{{\color{delim}{\{}}}}{1}
        {\}}{{{\color{delim}{\}}}}}{1}
        {[}{{{\color{delim}{[}}}}{1}
        {]}{{{\color{delim}{]}}}}{1},
}

\begin{lstlisting}[language=json,firstnumber=1,label={lst:model-json}]
[
    // Model identifier (as a byte array)
    [31, 77, 69, 112, ..., 125],
    // Model type
    "Acids",
    // Context specifier type
    "generic_ao4_qo1_pb2",
    [
        [
            // Context specifiers (represented as integers)
            [420, 2137, 2403],
            // Context
            [
                // Context probability
                // The sum of all context probabilities in a model should be 1
                0.1234,
                // Symbol probabilities
                // (in this case: N, A, C, T, G, respectively)
                // The sum of all symbol probabilities in a context should be 1
                [0.0, 0.2, 0.3, 0.4, 0.1]
            ]
        ],
        // ... more contexts
    ]
]
\end{lstlisting}

The \emph{model identifier} is an SHA-3\cite{1421} 256-bit checksum of the
entire model contents.
When deserializing the model from a file, the identifier indicates if the
model has been read correctly.
When reading a sequence file, the identifier list tells the decompressor
which models to use.

The identifier generation process starts with serialized by storing the model
type, context specifier type, model map sorted by keys ascending, and then
the contexts themselves.
Then, the hash of such a blob is calculated.
